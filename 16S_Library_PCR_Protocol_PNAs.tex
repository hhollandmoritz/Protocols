\documentclass[11pt]{article}

\usepackage[margin=1in]{geometry}
\usepackage{amssymb} %for low tilde
\usepackage{amsmath}
\usepackage{dirtytalk} %quotations
\usepackage{enumitem} %for continuing enumerated lists
\usepackage{fancyhdr}
\usepackage{gensymb} %for symbols such as \degree
\usepackage{graphicx}
\usepackage{hyperref} %for hyperlinks

\pagestyle{fancy}

\lhead{Hannah Holland-Moritz}
\chead{16S Library Protocol - PNA Blockers}
\rhead{\today}


\begin{document}

\begin{center}
\textbf{\large{16S Library PCR Protocol with PNA Blockers}}
\break

\emph{This protocol is a modified version of the \href{http://www.earthmicrobiome.org/emp-standard-protocols/16s/}{Earth Microbiome Project's PCR protocol} and the JGI's PNA PCR Protocol.}
\end{center}

\noindent{\textbf{Notes:}}
There are 25 each of forward and reverse primer barcodes that can be combined to make 625 total unique barcode combinations. The primers are at 10$\mu$M each and are mixed in equal parts to create the 10$\mu$M combinations. \\
\hfill
\\
We use the bacteria/archaeal primers 515F/806R with an inhouse barcode system designed by Aaron Darling. \\

\noindent{More about the primers can be found here:}\\
\\
\noindent{\href{http://www.nature.com/ismej/journal/v6/n8/full/ismej20128a.html}{Caporaso JG, Lauber CL, Walters WA, Berg-Lyons D, Huntley J, Fierer N, Owens SM, Betley J, Fraser L, Bauer M, Gormley N, Gilbert JA, Smith G, Knight R. 2012. Ultra-high-throughput microbial community analysis on the Illumina HiSeq and MiSeq platforms. ISME J.}}

\noindent{\rule{\textwidth}{1pt}} %draws line across page that stops at margins
\noindent{\textbf{1X Recipe:}} \\
\hfill
22.5 $\mu$L of Invitrogen Platinum SuperMix \\
\hfill
1.25 $\mu$L of Primer Mix (10$\mu$M)\\
\hfill
0.62 $\mu$L of PNA chloroplast blocker ($\sim50\mu$M)\\
\hfill
0.62 $\mu$L of PNA mitochondria blocker ($\sim50\mu$M)\\
\hfill
1.0-5.0 $\mu$L of template DNA
\\
\\
\noindent{\textbf{Total:} 26-30$\mu$L}
\\
\rule{\textwidth}{1pt} %draws line across page that stops at margins
\noindent{\textbf{Protocol:}}
\medskip \break
\noindent{\emph{PCR}}
\begin{enumerate}
\item Amplify samples in triplicate, meaning each sample will be amplified in 3 replicate 25 $\mu$L PCR reactions.
\item Combine the triplicate PCR reactions for each sample into a single volume. Combination will result in a total of 75 $\mu$L of amplicon for each sample. Do NOT combine amplicons from different samples at this point.
\item Run 5 $\mu$l of amplicons for each sample on an agarose gel to verify amplification. Expected band size for roughly 300-400 bp
\end{enumerate}
\noindent{\emph{Clean-up}}
\begin{enumerate}[resume]
\item{Clean PCR reactions with magnetic beads. See \say{Magnetic-bead PCR Cleaning Protocol} for an example protocol.}
\item Quantify amplicons with Qubit.
\end{enumerate}
\emph{Pooling}
\begin{enumerate}[resume]
\item Combine an equal amount of amplicon from each sample into a single, sterile tube. Generally 240 ng of DNA per sample are pooled. However, higher amounts can be used if the final pool will be gel-isolated or when working with low biomass samples.
\emph{Note:} When working with multiple plates of samples, it is typical to produce a single tube of amplicons for each plate of samples.
\item If spurious bands were present on gel (in step 3), half of the final pool can be run on a gel and then gel-extracted to select only the target bands.*
\\
\\
*We routinely see high molecular-weight bands or smears in the gel. We use the Pippin prep gel from Sage Science (\href{http://www.sagescience.com/products/pippin-prep/}{http://www.sagescience.com/products/pippin-prep/}) to select the correct band size.
\item Measure concentration and 260/280 of final pool that has been cleaned. For best results the 260/280 should be between 1.8-2.0.
\item Send an aliquot for sequencing along with sequencing primers.
\end{enumerate}
\rule{\textwidth}{1pt} %draws line across page that stops at margins
\textbf{Thermocycler Conditions:}
\\
\\
\emph{For 96 well thermocyclers:}
\begin{enumerate}
\item 94\degree C 3 minutes
\item 94\degree C 45 seconds
\item 50\degree C 60 seconds
\item 72\degree C 90 seconds
\item Repeat steps 2-4 35 times
\item 72\degree C 10 minutes
\item 4\degree C HOLD
\end{enumerate}
\emph{For 384 well thermocyclers:}
\begin{enumerate}
\item 94\degree C 3 minutes
\item 94\degree C 60 seconds
\item 50\degree C 60 seconds
\item 72\degree C 105 seconds
\item Repeat steps 2-4 35 times
\item 72\degree C 10 minutes
\item 4\degree C HOLD
\end{enumerate}


\end{document}
