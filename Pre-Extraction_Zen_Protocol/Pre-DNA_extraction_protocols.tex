%%%%%%%%%%%%%%%%%%%%%%%%%%%%%%%%%%%%%%%%%
% Contract
% LaTeX Template
% Version 1.0 (December 8 2014)
%
% This template has been downloaded from:
% http://www.LaTeXTemplates.com
%
% Original author:
% Brandon Fryslie
% With extensive modifications by:
% Vel (vel@latextemplates.com)
%
% License:
% CC BY-NC-SA 3.0 (http://creativecommons.org/licenses/by-nc-sa/3.0/)
%
% Note:
% If you are using Apple OS X, go into structure.tex and uncomment the font
% specifications for OS X and comment out the default specifications - this will
% drastically increase how good the document looks. You will now need to
% compile with XeLaTeX.
%
%%%%%%%%%%%%%%%%%%%%%%%%%%%%%%%%%%%%%%%%%

\documentclass[a4paper,12pt]{article} % The default font size is 12pt on A4 paper, change to 'usletter' for US Letter paper and adjust margins in structure.tex

%%%%%%%%%%%%%%%%%%%%%%%%%%%%%%%%%%%%%%%%%
% Contract
% Structural Definitions File
% Version 1.0 (December 8 2014)
%
% Created by:
% Vel (vel@latextemplates.com)
% 
% This file has been downloaded from:
% http://www.LaTeXTemplates.com
%
% License:
% CC BY-NC-SA 3.0 (http://creativecommons.org/licenses/by-nc-sa/3.0/)
%
%%%%%%%%%%%%%%%%%%%%%%%%%%%%%%%%%%%%%%%%%

%----------------------------------------------------------------------------------------
%	PARAGRAPH SPACING SPECIFICATIONS
%----------------------------------------------------------------------------------------

\setlength{\parindent}{0mm} % Don't indent paragraphs

\setlength{\parskip}{2.5mm} % Whitespace between paragraphs

%----------------------------------------------------------------------------------------
%	PAGE LAYOUT SPECIFICATIONS
%----------------------------------------------------------------------------------------

\usepackage{geometry} % Required to modify the page layout

\setlength{\textwidth}{16cm} % Width of the text on the page
\setlength{\textheight}{24.5cm} % Height of the text on the page

\setlength{\oddsidemargin}{0cm} % Width of the margin - negative to move text left, positive to move it right

% Uncomment for offset margins if the 'twoside' document class option is used
%\setlength{\evensidemargin}{-0.75cm} 
%\setlength{\oddsidemargin}{0.75cm}

\setlength{\topmargin}{-1.25cm} % Reduce the top margin

%----------------------------------------------------------------------------------------
%	FONT SPECIFICATIONS
%----------------------------------------------------------------------------------------

% If you are running Apple OS X, uncomment the next 4 lines and comment/delete the block below, you will now need to compile with XeLaTeX but your document will look much better

%\usepackage[cm-default]{fontspec}
%\usepackage{xunicode}

%\setsansfont[Mapping=tex-text,Scale=1.1]{Gill Sans}
%\setmainfont[Mapping=tex-text,Scale=1.0]{Hoefler Text}

%-------------------------------------------

\usepackage[utf8]{inputenc} % Required for including letters with accents
\usepackage[T1]{fontenc} % Use 8-bit encoding that has 256 glyphs

\usepackage{avant} % Use the Avantgarde font for headings
\usepackage{mathptmx} % Use the Adobe Times Roman as the default text font together with math symbols from the Sym­bol, Chancery and Com­puter Modern fonts

%----------------------------------------------------------------------------------------
%	SECTION TITLE SPECIFICATIONS
%----------------------------------------------------------------------------------------

\usepackage{titlesec} % Required for modifying section titles

\titleformat{\section} % Customize the \section{} section title
{\sffamily\large\bfseries} % Title font customizations
{\thesection} % Section number
{16pt} % Whitespace between the number and title
{\large} % Title font size
\titlespacing*{\section}{0mm}{7mm}{0mm} % Left, top and bottom spacing around the title

\titleformat{\subsection} % Customize the \subsection{} section title
{\sffamily\normalsize\bfseries} % Title font customizations
{\thesubsection} % Subsection number
{16pt} % Whitespace between the number and title
{\normalsize} % Title font size
\titlespacing*{\subsection}{0mm}{5mm}{0mm} % Left, top and bottom spacing around the title % Input the structure.tex file which specifies the document layout and style

\usepackage{graphicx} %for pictures
\usepackage{gensymb} %for degree symbol


%----------------------------------------------------------------------------------------

\begin{document}

\begin{center} \textbf{Preparation of Zen Seagrass Samples for DNA Extraction:} \end{center}

All initial transfers of samples into the MoBio bead-beating tubes should be performed on the bench top in the presence of a flame.


%----------------------------------------------------------------------------------------
%	OBJECTIVE SECTION
%----------------------------------------------------------------------------------------

\section{Filters}

The filters are cut in half and placed in two separate tubes. Each tube is labelled identically to the other. In each tube, the filter is folded in half multiple times such that it forms a vague triangle shape from the center of the round filter to the edge. These triangles are often folded in half once more to make trapezoids to aid fitting into the tube.
 
\includegraphics[scale=0.9]{filterdiagram}

\textbf{Vortex sample tube for 30 seconds}, to help remove any organisms stuck to the outside of the filter into the liquid. Use \textbf{ethanol-flame sterilized} forceps and scissors to unfold and cut the \textbf{tip of each filter} half (from the two identically-labeled tubes) into the MoBio powersoil tube. \textbf{Resterilize the scissors and forceps with ethanol and flame} between filters. Using a pipette set between 250 and 500 ul transfer liquid into the MoBio powersoil tube until it is no more than \textbf{2/3 full.} If you have to compromise between putting in liquid and putting in filter tips (because tube is more than 2/3 full), liquid is preferable.

%----------------------------------------------------------------------------------------
%	DELIVERABLES SECTION
%----------------------------------------------------------------------------------------

\section{Soil}

\textbf{Vortex sample tube for 30 seconds,} to insure sample is well homogenized before transfer. As carefully as possible, so as not to spill sample, use a small metal \textbf{ethanol-flame sterilized} scoopula or spatula to transfer sample into MoBio powersoil tube. Fill tube no more than \textbf{2/3 full} (so there is room for C1). \textbf{Rinse spatula/scoopula in small beaker of water} to remove visible sediment, then dip in \textbf{ethanol to flame sterilize} once more.

%----------------------------------------------------------------------------------------
%	SCHEDULE SECTION
%----------------------------------------------------------------------------------------

\section{Leaf}

\textbf{Vortex sample tube for 30 seconds,} to help remove any organisms stuck to the outside of the leaf into the liquid. Using a pipette set between 250 and 500 ul transfer as much liquid as possible into the MoBio powersoil tube until it is no more than \textbf{2/3 full} or all the liquid is used up.

%----------------------------------------------------------------------------------------
%	TERMS AND CONDITIONS SECTION
%----------------------------------------------------------------------------------------

\section{Root}

\textbf{Vortex sample tube for 30 seconds,} to help remove any organisms stuck to the outside of the roots into the liquid. Using a pipette set between 250 and 500 ul transfer as much liquid as possible into the MoBio powersoil tube until it is no more than \textbf{2/3 full} or all the liquid is used up.

%------------------------------------------------

**Filled MoBio powersoil tubes \textbf{may be stored at -20\degree C for up to 24 hours} before adding C1**

%------------------------------------------------


%----------------------------------------------------------------------------------------

\end{document}